\chapter{Introduction}
\label{intro}
Repetitive DNA makes up a significant portion of eukaryotic organisms \cite{britten1968repeated}. As a result, the detection of repetitive sequences of DNA has become a significant problem in bioinformatics \cite{pevzner2004de-novo}. A repeat finding tool should be able to detect both identical and approximate repeats, the latter being recurring subsequences that are extremely similar but not identical. This proposal seeks to improve an already existing repeat finding tool, known as Rapid Ab Initio Detection of Elementary repeats (RAIDER) \cite{figueroa2013raider}. Through an in-depth study of this tool and of approaches to finding approximate repeats, awe hope to significantly improve upon RAIDER's current ability to detect approximate repeats.

Over long periods of time, originally identical repetitive sequences, which were initially all copies of some ancestral sequence, have accumulated mutations. Therefore, the detection of approximate repeats can provide information about a particular genome as well as the composition of genomes from previous generations. Thus, a repeat finding algorithm's ability to detect approximate repeats is extremely valuable.

In this proposal, we will review the necessary background associated with the approximate repeat-finding problem. This will include a cursory discussion of the genome and DNA as strings, followed by a more in depth discussion of concepts such as string matching, as well as the characterization of both exact (identical) repeats and approximate (similar) repeats. 

Following a review of the common techniques employed in repeat-finding tools, we will focus on background and definitions associated with $k$-mer and spaced seed approaches, which form the basis for the RAIDER algorithm. Lastly, we will discuss our proposed study of improving RAIDER's sensitivity to finding approximate repeats and give an expected timeline of our work.

\clearpage