\section{Lmer (k-Mer) Approach}
One approach to repeat-finding that is of particular interest is the $k$-mer approach, which will henceforth be referred to as the \textit{Lmer} approach. An Lmer is simply a subsequence of the genomic sequence of length equal to $L$, some fixed constant. We will begin to describe this topic using some observations and definitions from Figueroa \cite{figueroa2013raider}.

\begin{lem}
Every Lmer belongs to at most one elementary repeat family.
\end{lem}

\begin{defn}
Given strings $x$ and $y$, define \textbf{x$\circ$y} as the string $abc$, where $x=ab$, $y=bc$, and $b$ is the longest substring that is both a suffic of $x$ and a prefix of $y$.
\end{defn}

\begin{defn}[Lmer Series]
An \textbf{Lmer series} is a sequence of Lmers $x_{0}, x_{1}, \dotsc x_{k-1}$ in the query sequence such that the start coordinate of $x_{i}$ is one greater than the start coordinate of $x_{i-1}$. A sequence $F$ is \textbf{composed} from an Lmer series if $F=x_{0}\circ x_{1}\circ\dotsm x_{k-1}.$
\end{defn}

\begin{defn}[Lmer Decomposition]
Given a subsequence $F$ of the query sequence, the \textbf{Lmer decomposition} of $F$ is the Lmer series $x_{0}, x_{1}, \dotsc x_{k-1}$ such that $F$ is composed of the series.
\end{defn}

Using these definitions, we can go on to redefine elementary repeats using Lmers.