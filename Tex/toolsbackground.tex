
\section{Survey of Repeat Finding Tools}
Over the years following the discovery of the abundance of repetitive DNA, a variety of repeat finding algorithmic approaches and tools have been developed \cite{saha2008computational}. Repeat finding tools can be broken down into two main categories: library-based and ab initio.

\subsubsection{Library Based Tools. } Library-based tools compare input genomic data to an existing library of repeat sequences in order to identify known repeats. RepeatMasker \cite{smit1996repeatmasker} is currently the most widely used library-based repeat finding tool \cite{saha2008computational}. It compares the consensus sequences from known repeat families, stored in a database called RepBase, to search for new members of each family based on similarity.

\subsubsection{Ab Initio Tools. }Ab initio tools attempt to identify repeats without using any pre-existing knowledge of known repeat sequences or motifs. While both of these techniques are widely used, \textit{ab initio} tools are becoming increasingly important due to the rise in number and diversity of sequences coming from genome sequencing projects. 

The approaches that have been used so far in the ab initio identification of repeats and are of interest to this research can be grouped into two basic categories \cite{saha2008computational}.

\begin{enumerate}
\item{k-mer approaches find all exact substrings that have a frequency equal to or greater than some defined threshold. Examples of tools that are based on this approach include REPuter \cite{kurtz2001reputer} and RepeatScout \cite{price2005novo}.}
\item{Spaced seed approaches are similar to k-mer approaches, but they use spaced seeds when matching two strings in order to allow some predefined amount of differences between the strings (including substitutions, insertions, and deletions). PatternHunter \cite{ma2002patternhunter} was the first tool to use this approach. It was later followed by PatternHunter II \cite{li2003patternhunter2}, which allowed for the use of multiple spaced seeds.}
\end{enumerate}




