\subsection{Biological Background}
Every living organism inherits hereditary information from its parents that affect the organism's distinguishing traits. This information is embedded inside an organism's genetic material, a molecule known as deoxyribonucleic acid (DNA). An organism's \textit{genome} is the set of all DNA sequences associated with that organism \cite{lewin2014lewins}.

Each sequence of DNA is composed of a chain of nucleotides. There are four nucleotides found in DNA: Adenine (A), Cytosine (C), Guanine (G), and Thymine (T). DNA can therefore be represented as a finite string $s=s_{0}s_{1}...s{n-1}$ over the alphabet $\Sigma=\lbrace A, C, G, T\rbrace$ of nucleotides \cite{elloumi2011algorithms}.

\textit{Repetitive DNA} is defined as a set of discrete DNA sequences in the same genome that are similar or identical to one another \cite{treangen2012repetitive}. As previously mentioned, it makes up a significant portion of most genomes, especially in eukaryotic organisms. 

\begin{defn}
Let $F$ be a subsequence of the query sequence with Lmer decomposition $x_{1},x_{2},\dotsc ,x_{k}$, where $k=\lvert F \rvert - L + 1$. $F$ is an \textbf{elementary repeat} if:
\begin{enumerate}
\item $k \geq 1$
\item $freq(F) \geq f$
\item $freq(x_{i})=freq(F)$ for all Lmers $x_{i}$ in the decomposition
\item $k$ is maximal. That is, there is no Lmer $y$ such that $y\circ F$ or $F\circ y$ meets conditions 1-3
\end{enumerate} \cite{figueroa2013raider}.
\end{defn}



