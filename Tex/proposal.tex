\chapter{Proposed Research}
\label{proposal}
As previously mentioned, we propose to do an in-depth study of a pre-existing tool, RAIDER \cite{figueroa2013raiderpaper}, in order to both inspect and improve upon its current ability to detect approximate repeats. Many of the established and emerging repeat-finding tools seem to employ the use of $k$-mers, spaced seeds, or both in order to most effectively detect repeats \cite{saha2008computational}. Additionally, RAIDER is currently employing both of these approaches in its algorithmic approach. For these reasons, we will start this study by trying to use a fundamentally similar approach to RAIDER and investigating how to make RAIDER better use spaced seeds. Additionally, we will look into how to make RAIDER best use multiple spaced seeds, as is used in PatternHunter II \cite{li2003patternhunter2}.

We will also look into seed design and analysis, attempting to determine what kind of spaced seed is "best" for repeat-finding purposes. An  optimized spaced seed design could allow for a few optimized spaced seeds to be as sensitive to repeat finding as many non-optimized spaced seeds. This kind of information could allow for an improved detection of repeats without significantly affecting the space complexity of the RAIDER algorithm. 

Along the way, it may become evident that the best way to improve RAIDER's detection of approximate repeats will be to pivot away from the use of spaced seeds in favor of some other approach. We remain open to this idea, especially since it seems that spaced seeds are limited in that they can only be used to determine $k$-mismatches approximate repeats, not $k$-differences approximate repeats.