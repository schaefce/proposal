\documentclass[12pt]{article}
\usepackage[margin=1in]{geometry}
\usepackage{titlesec}


\usepackage{enumitem}
\setlist{itemsep=.5pt,topsep=2pt,parsep=2pt}
%\onehalfspacing


\titleformat{\section}[runin]{\normalsize\bf}{}{}{}[]
\titlespacing*{\section}{0pt}{5pt}{2\parsep}

\begin{document}

\section*{Prompt}
Please outline your educational and professional development plans and career goals. How do you envision graduate school preparing you for a career that allows you to contribute to expanding scientific understanding as well as broadly benefit society? Page limit - 3 pages 

Describe your personal, educational and/or professional experiences that motivate your decision to pursue advanced study in science, technology, engineering or mathematics (STEM). Include specific examples of any research and/or professional activities in which you have participated. Present a concise description of the activities, highlight the results and discuss how these activities have prepared you to seek a graduate degree. Specify your role in the activity including the extent to which you worked independently and/or as part of a team. Describe the contributions of your activity to advancing knowledge in STEM fields as well as the potential for broader societal impacts (See Solicitation, Section VI, for more information about Broader Impacts). 

NSF Fellows are expected to become globally engaged knowledge experts and leaders who can contribute significantly to research, education, and innovations in science and engineering. The purpose of this statement is to demonstrate your potential to satisfy this requirement. Your ideas and examples do not have to be confined necessarily to the discipline that you have chosen to pursue.

\clearpage
\section*{Academic Background}
Over my years at Miami University, I explored a variety of majors and fields. My first major, declared as a freshman, was mathematics. I enjoyed high-level mathematics and statistics courses, while also exploring a variety of subjects such as economics and organic chemistry. I also took numerous courses in psychology, which unintentionally led to my completion of a second major in psychology. Throughout my exploration of these diverse subjects, I have been most challenged and stimulated by those that required strong analytic thinking and problem solving. For that reason, I continued to take courses in mathematics and statistics even after I had completed the requirements for my major.

I had not had any experience with programming prior to taking a mandatory computer science course at the end of my senior year, in the spring of 2013. I took an introductory class for beginning computer science and engineering students. Taking this class changed the course my academic life. Although I had enjoyed many courses, I had never before felt such a passion for what I was learning. My professor suggested that I take a few more courses in the summer to further evaluate my interest and aptitude, and then to consider applying to graduate school. I took his advice, deferred graduation and took two more computer science courses, Object-Oriented Programing and Data Structures, over the summer.  My interest in the subject had only grown. My intention at the end of the summer was to get a minor in computer science, and apply to graduate school in computer science. 

I spoke with the Graduate Director about  which computer science courses I should be taking to prepare myself for applying to graduate school. He recommended applying to the combined program, as that would make more sense for someone in my particular circumstances. So, I spent the 2013-2014 academic year taking graduate courses in computer science. 


These graduate courses were challenging, especially those that were more coding intensive. I felt like I was behind in comparison to the other students. I worked twice as hard to make up for this. Luckily, I started off also taking a few theory-based courses such as Algorithms and Automata, which were easier for me because they were strongly linked to my background in mathematics. I think that my experience in the Masters of Computer Science program would equate to an accelerated undergraduate program, as I learned a lot of background information in a very short period of time. Additionally, my work for my thesis has given me a lot of experience in computer science research.

\section*{Thesis Research}


\section*{Graduate Assistantship}
My strong mathematical background gave me an advantage when I took theory-based computer science courses. For this reason, I was asked to be the teaching assistant for the Algorithms course professor during my second semester as a graduate student. I was in charge of grading problem sets, as well as holding office hours to provide additional help for students. Due to positive reviews from the professor I was assisting, the Department Head asked me to be the teaching assistant for Computer Architecture. Additionally, I was asked to tutor an undergraduate student in the introductory Software Engineering course.

Upon the suggestion of my research advisor, I applied for and received a Graduate Assistantship for the 2014-2015 academic year. This award covers my tuition and gives me a living stipend in exchange for my work as a teaching assistant for a professor. This semester, I am the teaching assistant for Algorithms, Data Structures, and Operating Systems.

As a teaching assistant, I have experience grading papers, problem sets, and proofs, as well as holding office hours to provide additional help for undergraduate and graduate students in various courses. 

\section*{Amazon Internship}



My research, teaching, and leadership experiences have shaped my desire to pursue a PhD and become a professor of computer science. While I am determined to satisfy these goals regardless of attaining the NSF GRF, acquiring such a tremendous fellowship will increase the quality and effectiveness of my research in graduate school.


\end{document}
