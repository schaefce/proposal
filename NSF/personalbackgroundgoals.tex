\documentclass[12pt]{article}
\usepackage[margin=1in]{geometry}
\usepackage{titlesec}
\usepackage[numbers,compress]{natbib}
\usepackage{todonotes}
\usepackage{enumitem}
\usepackage{amsmath}
%\linespread{2}
\setlist{itemsep=.5pt,topsep=2pt,parsep=1pt}
\def\bibfont{\footnotesize}


\titleformat{\section}[hang]{\centering\normalsize\bf}{}{}{}[]
\titlespacing*{\section}{0pt}{4pt}{0pt}
\titleformat{\subsection}[runin]{\normalsize\bf}{\thesubsection}{10pt}{}[]
\titlespacing{\subsection}{0pt}{2pt}{1.5\parsep}
\renewcommand{\thesubsection}{1.\arabic{subsection}}

\begin{document}

%\section*{Prompt}
%Please outline your educational and professional development plans and career goals. How do you envision graduate school preparing you for a career that allows you to contribute to expanding scientific understanding as well as broadly benefit society? Page limit - 3 pages 

%Describe your personal, educational and/or professional experiences that motivate your decision to pursue advanced study in science, technology, engineering or mathematics (STEM). Include specific examples of any research and/or professional activities in which you have participated. Present a concise description of the activities, highlight the results and discuss how these activities have prepared you to seek a graduate degree. Specify your role in the activity including the extent to which you worked independently and/or as part of a team. Describe the contributions of your activity to advancing knowledge in STEM fields as well as the potential for broader societal impacts (See Solicitation, Section VI, for more information about Broader Impacts). 

%NSF Fellows are expected to become globally engaged knowledge experts and leaders who can contribute significantly to research, education, and innovations in science and engineering. The purpose of this statement is to demonstrate your potential to satisfy this requirement. Your ideas and examples do not have to be confined necessarily to the discipline that you have chosen to pursue.

%\clearpage
When I was three years old, I climbed up to the top of my swim club's 10 foot diving board, and I jumped off. 
You might be wondering if I was an advanced diver for my young age; I wasn't. 
The jump ended in a painful belly flop, followed by a long lecture from my relieved, yet angry, mother. 
Looking back, it probably wasn't the best decision I've made. 
So why did I do it? Because I believed I could.

For as long as I can remember, I have had this deep-set belief that I could do anything. 
Over the years, this has led me to set and achieve high goals without ever questioning whether they were possible.
My goals have gotten harder to reach over the years, but I still strongly believe that nothing is impossible for those who are willing to put in the necessary time and work.
This approach to problem-solving is the basis for much of my success, and I think that it will continue to ensure my success in graduate school and beyond.

My belief that almost anything can be achieved with enough hard work is actually what drives my desire to become a research professor. 
Research centers around finding a problem to study and then working diligently to solve that problem.
The idea of working relentlessly on a problem actually excites me, because I am not afraid of working hard and the possibility of discovering something new is extremely motivating for me.
Additionally, I would like to be a professor because I want the opportunity to help students realize that they have the capability and the potential to solve any problem they encounter.
I can't think of a better way to give back to my field than convincing at least a few young minds to believe in themselves and to not give up so easily when faced with a problem.


\section*{Relevant Background}
\subsection{Academics}
Over my years at Miami University, I explored a variety of majors and fields. 
My first major, declared as a freshman, was Mathematics. 
I enjoyed high-level Mathematics and Statistics courses, while also exploring subjects such as Economics and Organic Chemistry. 
I also took numerous courses in Psychology, unintentionally leading to my completion of a second major.
Throughout my exploration of these diverse subjects, I have been most challenged and stimulated by those that required strong analytic thinking and problem solving. 
For that reason, I continued to take courses in Mathematics and Statistics even after completing the requirements for my major.

I had no experience with programming prior to taking a mandatory Computer Science course at the end of my senior year, in the spring of 2013. 
%I took an introductory class for beginning Computer Science and Engineering students. 
Taking this class changed the course of my academic life. 
Although I had enjoyed many courses, I had never before felt such a passion for what I was learning. 
My professor suggested that I take a few summer courses to further evaluate my interest and aptitude, and then to consider applying to graduate school.
I took his advice, deferred graduation, and took two Computer Science courses over the summer.  
My interest in the subject had only grown;
my intention at the end of the summer was to get a minor in Computer Science and apply to graduate school. 

I spoke with the Graduate Director about which computer science courses I should take to prepare for applying to graduate school; 
he believed that applying to Miami's combined BS/MS program would make more sense for me. 
So, I spent the 2013-2014 academic year taking graduate courses in Computer Science. 
These courses were challenging, especially those that were more coding intensive. 
I felt like I was behind in comparison to the others, so I worked twice as hard to make up for it. 
%Luckily, I started off also taking a few theory-based courses such as Algorithms and Automata, which were easier for me because they were strongly linked to my mathematics background.
I think that my experience in the Masters of Computer Science program would equate to an accelerated undergraduate program, as I learned a lot of background information in a very short period of time.
%Additionally, my thesis work has given me experience in computer science research.

\subsection{Teaching}
My strong mathematical background gave me an advantage when I took theory-based computer science courses. 
For this reason, I was asked to be the teaching assistant for the Algorithms course professor during my second semester in the Computer Science department.% as a graduate student. 
I was in charge of grading problem sets, as well as holding office hours to provide additional help for students. 
Due to positive reviews from the professor I was assisting, the department head asked me to be the teaching assistant for Computer Architecture, as well as tutor an undergraduate student in the introductory Software Engineering course.

I applied for and received a Graduate Assistantship for the 2014-2015 academic year. 
This award covers my tuition and gives me a living stipend in exchange for my work as a teaching assistant for a professor (as Miami has no PhD program, there are few research assistantships; Dr. Karro has told me I am first in line if one becomes available). 
This semester, I am the teaching assistant for Algorithms, Data Structures, and Operating Systems. 

Algorithms was one of my favorite computer science courses, and my love for the subject has only grown from being the teaching assistant for it.
I respect students who are motivated enough to attend my office hours, so I try to be as prepared as they are to discuss the assignment.
Additionally, I try to keep an open mind when students ask me questions about the validity of their approach, so as to not discount a student's solution just because it is not the same as mine.
As a result, I spend a significant portion of office hours at the whiteboard with students, walking through their solution until we are both satisfied that it is correct.
This benefits me as much as, if not more than, the students because it allows me to consider a wide variety of solutions to a problem, many of which I had not considered.

\subsection{Research}
For the past year, I have been working with Dr. John Karro, an Associate Professor at Miami University. 
I have been investigating the use of spaced seeds in order to improve the sensitivity of a repeat-finding tool created in our lab, known as RAIDER \cite{figueroa2013raiderpaper}. 

In order to make statements about whether my changes significantly improved RAIDER, I needed a way to quantitatively evaluate the results of the tool. 
I worked with Dr. Karro to create an evaluation tool that would compare the results of RAIDER and RepeatScout \cite{price2005novo}, a widely used repeat-finding tool, when both were run on the same simulated DNA sequence. 
%Each run is time intensive, so I worked to automate the test runs and create a pipeline to have them run in parallel where possible.
After getting the results of both tools, I created a module to calculate relevant statistics about their performance, including the sensitivity and specificity for each tool.

After completing this evaluation tool, I started working on generalizing RAIDER to use spaced seeds in order to find approximate repeats.
I have been investigating how to design an ``optimized'' spaced seed for repeat-finding purposes, as well as how to modify the RAIDER algorithm to allow for the use of multiple spaced seeds.
This approach has shown to have significant effects on both sensitivity and speed in homology search tools \cite{ma2002patternhunter}, so we believe that the use of spaced seeds could also allow for improved sensitivity in repeat-finding tools.

\section*{Future Goals}
I have two main motivations for pursuing a Ph.D. and becoming a research professor.
The first is the research: the idea of finding a problem to focus on and work relentlessly to solve sounds fun to me.
The possibility of discovering something new, perhaps even something thought to be impossible, excites me.
Further, I enjoy implementing and experimentally verifying my solutions to prove their correctness (both in a formal sense, and in the context of laying out convincing evidence in the context of a paper or conference presentation).
As someone who loves challenges, I can't imagine a more fulfilling job than trying to solve new and interesting problems in my field.

The second motivation I have for becoming a research professor is the potential to influence students in my field.
I have seen many students who see a problem and almost immediately give up, and this behavior used to puzzle me - how could they just give up on something they had barely even started?
I have since discovered that many students don't believe that they will ever be able to solve the problem, no matter how much effort they put forth, because they lack confidence in their own abilities.
I believe that having the chance to convince students to not give up so easily on themselves would be extremely rewarding.

\section*{Intellectual Merit}
I have worked extremely hard during my time as both an undergraduate and a graduate student at Miami University. 
Academically, I have performed very well in both mathematics and computer science. 
This performance led me to be selected as a teaching assistant for Algorithms, which was an honor as it is one of the most difficult computer science courses and showed that the professor trusted my competency as both a former student and as a grader.
This position was and still is a great opportunity for me to help students to better understand difficult problems.
Additionally, I have experience performing graduate-level research in the field of bioinformatics. 

\section*{Broader Impacts}
As a research professor, I believe that I would have a significant positive effect on the fields of Computer Science and Bioinformatics.
I believe that I would contribute to the advancement of these fields through my research, because I am a diligent worker and I tend not to limit myself in terms of what others think is possible.
I am willing to take risks and attempt to solve problems that are commonly avoided due to their perceived complexity, which I think would make my work meaningful and possibly motivate the work of others in these fields.

I also want to be a research professor so I can positively influence my student advisees and pass on to them the ability to approach all problems with confidence and tenacity. 
Further, I would like to serve as a role model to female students interested in Computer Science. 
While taking a Social Psychology course a few years ago, I found one study to be especially interesting- the majority of individuals implicitly associate science with males than with females. 
This actually surprised me, because I never felt threatened or at a disadvantage as a female in the field of mathematics or science. 
One of the reasons for this, I suspect, is because both my mother and my father have strong mathematical backgrounds. 
It seems likely that gender stereotypes are less likely to negatively affect people who are exposed to a strong counterexample of the stereotype, 
so I would be more than happy to serve as that counterexample for female students in Computer Science. 
I would make it my goal to encourage all of my students, both male and female, to believe that they can do anything if they try hard enough.

\begingroup
    \setlength{\bibsep}{-1pt}
    \bibliography{bibliography}
    \bibliographystyle{myIEEEtran}
\endgroup
\end{document}
