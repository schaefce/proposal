\documentclass[12pt]{article}
\usepackage[margin=1in]{geometry}
\usepackage{titlesec}
\usepackage[numbers,compress]{natbib}
\usepackage{todonotes}
\usepackage{enumitem}
\usepackage{amsmath}
%\linespread{2}
\setlist{itemsep=.5pt,topsep=2pt,parsep=1pt}
\def\bibfont{\footnotesize}


\titleformat{\section}[runin]{\normalsize\bf}{}{}{}[]
\titlespacing*{\section}{0pt}{4pt}{1.5\parsep}


\begin{document}
%\section*{Prompt}Present an original research topic that you would like to pursue in graduate school. Describe the research idea, your general approach, as well as any unique resources that may be needed for accomplishing the research goal (i.e., access to national facilities or collections, collaborations, overseas work, etc.) You may choose to include important literature citations. Address the potential of the research to advance knowledge and understanding within science as well as the potential for broader impacts on society. The research discussed must be in a field listed in the Solicitation (Section X, Fields of Study).
%\clearpage


\section*{Abstract} 
I propose to use a spaced-seed search technique to develop an ab initio repeat-finding tool with improved sensitivity and run time relative to currently available tools.

\section*{Background}
Repetitive DNA sequences are sequences that appear multiple times, with possible variations, in a single genome \cite{lander2001initial}. 
They make up a significant portion of eukaryotic genomes, accounting for over 50\% of the human genome.
Repetitive DNA can create ambiguities in genome analysis, leading to incorrect interpretations of the results \cite{treangen2012repetitive}. 
Additionally, its analysis provides important insights when studying patterns of genomic evolution. 
As a result, the detection of repetitive DNA has become an important challenge in bioinformatics.

A variety of repeat-finding tools have been developed in the years following the discovery of repetitive DNA \cite{saha2008computational}. 
Library-based tools identify known repeats through comparison of input genomic data to an existing library of repeat sequences, while ab initio tools identify repeats without such knowledge. 
Because only ab initio tools can deal with newly sequenced genomes or discovering new repeat families, they are becoming increasingly important due to the rise in number and diversity of sequences coming from genome sequencing projects. 

%Repeats can be categorized to be either exact or approximate. 
Over long periods of time, originally identical repetitive sequences have accumulated mutations, creating approximate repeats - repeats that are still similar but no longer identical. 
%Approximate repeats can provide information about a particular genome as well as the genomic evolutionary history. 
%Thus, a repeat-finding tool's ability to detect approximate repeats is extremely valuable. 
This variation makes approximate repeats not only considerably harder to find, but also more valuable in that they contain information about the patterns of genomic evolution.



Beyond repeat identification, there lies a problem regarding how to best represent repeats. 
It has been suggested that repeats should be
%Pevzner, Tang, and Tesler \cite{pevzner2004de-novo} proposed that repeats could be 
represented as ``mosaics of subrepeats'', the claim being that this representation conserves evolutionary relationships between sub-repeats, which are often lost in other repeat representations \cite{pevzner2004de-novo}. This repeat representation problem should be kept in mind when designing new ab initio tools.
%\todo{This repeat representation problem needs to be addressed by new ab initio tools.}

\section*{Motivation}
I propose that ab initio search methods could be improved in terms of their sensitivity to approximate repeats through the use of the spaced seed matching technique that has proved successful in improving results from the BLAST tool.


Spaced seeds are patterns describing which positions in two strings  must match, and which positions are not so constrained, in order for them to be considered a ``valid'' match. 
A spaced seed $\pi$ is inherently defined by an ordered list of matching positions $M_{\pi} = \lbrace i_{1} \dotsc i_{w} \rbrace$. 
%The number of matching positions is the seed's weight, $w_{\pi}$.
%, and its length is denoted $\lvert \pi \rvert$. 
%The number of unconstrained positions (equal to $\lvert \pi \rvert - w_{\pi}$) limits the number of mismatches allowed. 
Two sequences $q$ and $t$ match with respect to $\pi$ if $q_{i} = t_{i} \ \forall i \in M_{\pi}$. 
When trying to find similar sequences in the context of homology search (e.g. the BLAST tool), it has been found that allowing tolerance to these mismatches over short spans considerably improves performance and result quality as compared to searching for short exact matches \cite{ma2002patternhunter}.
%Spaced seeds can be used to place an upper bound on the allowed difference between two sequences. 

%Two sequences can differ to some degree and still be a ``valid'' approximate match.
%One way we can quantify this difference is known as the Hamming distance, which is the minimum number of substitutions needed to transform the first sequence into the second \cite{kruskal1983an-overview}. 
%If we measure difference using Hamming distance, spaced seeds can be used to place an upper bound on the allowed difference between two sequences. 
%Therefore, it seems reasonable that spaced seeds could be used in ab initio search methods to find approximate repeats.

%For the past year, I have been doing research under Dr. John Karro, 
With Dr. John Karro, I have been
exploring the use of spaced seeds in order to improve the sensitivity of a repeat-finding tool created in our lab, RAIDER \cite{figueroa2013raiderpaper}. 
This tool only works on consecutive matches, and needs to be generalized to use spaced seeds.
%even use single spaced seeds.%only allows for a single, arbitrarily chosen spaced seed to be used in repeat identification. 
I have been revising the RAIDER's underlying algorithm
to allow for the use of single/multiple spaced seed sets, as well as investigating which characteristics make a seed better suited for repeat-finding purposes. 
This approach has had significant effects on both sensitivity and speed in homology search tools \cite{saha2008computational, ma2002patternhunter}. 
Although the associated challenges differ, it seems reasonable that an optimized spaced seed design could also improve sensitivity in repeat-finding tools.
%without significantly affecting time or space complexity in the space of repeat finding.


\section*{Proposed Research}
%The detection of repetitive sequences of DNA is a significant problem in bioinformatics, especially due to the increase in the number and diversity of genome sequences discovered via genome sequencing projects. 
The ability to quickly detect repetitive DNA is necessary in genome analysis, and becomes more so as the number of sequenced genomes continues to increase.
%There are three main criteria in which current ab intio repeat-finding tools are often lacking:
%I have found that 
Ab initio repeat-finding tools are often lacking in the following areas:
\begin{enumerate}
\item{\textit{Approximate repeat detection.} Many tools are unable to find approximate repeats without significantly more computational resources, but  finding only exact repeats results in obtaining a very small fraction of all repeat elements.}
\item{\textit{Time efficiency.} Some approaches can accurately detect repeats, but require quadratic time and space. 
This leads to an unacceptably long runtime, given the large size of genomic sequences.}
\item{\textit{Meaningful representation of output.} The evolutionary relationships between sub-repeats are often lost in representations of repeats by repeat-finding tools \cite{pevzner2004de-novo}. }
\end{enumerate}
I propose that ab initio approximate repeat-finding could be improved, while maintaining a reasonable runtime, through the use of a spaced-seed search technique. 
%I would like to further investigate repeat classification using this approach 
I would like to investigate this approach further
in order to develop an ab initio repeat-finding tool that meets all of the aforementioned criteria. 
My prior research involving repeat detection, in addition to my background in mathematics, will aid me in this research. 


%Intellectual Impact: The potential to advance knowledge. What will we learn as a result of your research.
\section*{Intellectual Merit}
I am proposing a spaced-seed approach to ab initio approximate repeat-finding that may significantly improve the detection of approximate repeats. 
Additionally, many repeat identification tools are based on sequence alignment, which is inherently slow. 
My approach is an alignment-free method, which would be significantly faster 
%and could have plenty of other applications. 
and could be applied to numerous other problems (e.g. searching meta-genomic datasets for common patterns that may indicate interspecies conservation, or, once available, large personal-genome datasets for recurring patterns that might indicate recurring abnormalities).


%Broader Impact:The potential to benefit society and contribute to the achievement of specific, desired societal outcomes. What practical effect will your work have on the lives of others.  (Where "others" is going to be mostly researchers who will have an easier time doing their research because of the tools you developed for their use.)
\section*{Broader Impacts}
This research could improve genome analysis and contribute to the study of molecular evolution, providing new tools for genome analysis that would be of direct use to researchers in the field.  
With the ability to quickly identify repeats in newly sequenced genomes, researchers looking to annotate genes % and other functional regions 
will be able to quickly filter repetitive areas, while those studying molecular evolution will have more data with which to study features such as genomic substitution rates \cite{treangen2012repetitive, karro2008exponential}. 
Additionally, the ab initio aspect of the tool allows for discovery of new repeat families, which will aid in the furthering our understanding of the evolutionary process and genomic structure at the molecular level. 
In short, this research will provide tools of value to those conducting genomic analysis, leading to an improved understanding of the human genome and corresponding benefits to human health.

\begingroup
    \setlength{\bibsep}{-1pt}
    \bibliography{bibliography}
    \bibliographystyle{abbrv}
\endgroup
\end{document}