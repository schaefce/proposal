\section{Thesis Research}
For the past year, I have been researching under Dr. John Karro, a professor in the Computer Science Department at Miami University whose main research area is bioinformatics. One significant problem in bioinformatics is the detection of repetitive DNA in a genome. My research seeks to improve the sensitivity of an already existing repeat-finding tool, known as Rapid Ab Initio Detection of Elementary Repeats (RAIDER) \cite{figueroa2013raiderpaper}.

Repetitive DNA is defined as a set of discrete DNA sequences in the same genome that are similar or identical to one another \cite{treangen2012repetitive}. It makes up a significant portion of most genomes, accounting for over one-third of the genetic material of higher organisms \cite{britten1968repeated}. Therefore, improving the effectiveness of a repeat-finding tool is of great value to the fields of bioinformatics and genomics.

\subsection{Literature Review}
Because the goal of my research is to ultimately improve RAIDER, my first major task was to completely understand RAIDER. I spent the first month or so reading Nathan's thesis and working through it until I had a deep understanding of the approach and the correctness of the algorithm.  Additionally, I researched other repeat-finding tools in order to understand how other repeat-finding approaches compare to RAIDER. The idea behind this review is that a thorough understanding of the current state of the art in repeat-finding tools will reveal various approaches on how to best modify and improve RAIDER.

Through this literature review, I gained experience reading, summarizing, and categorizing various sources in order to organize a knowledge base for a particular topic. A doctoral dissertation requires the review of a multitude of background sources, so being able to organize information efficiently is pivotal when seeking a graduate degree.

\subsection{Tool Evaluation}
In order to make any statements about whether my approach improved RAIDER, I needed a way to quantitatively evaluate RAIDER. I worked with Dr. Karro to create an evaluation tool that would compare the results of RAIDER and RepeatScout, a widely used repeat-finding tool, when both were run on the same simulated DNA sequence. This chromosome sequence simulator was created by Dr. Karro. I provided for the necessary steps in order to run RAIDER and RepeatScout on the simulated sequence, using a pipeline to continuously submit jobs to the Miami University cluster and read the results from these submitted jobs. The use of a cluster was necessary in order to quickly obtain the results of these tools.

After getting the results of both RAIDER and RepeatScout, I created a module to calculate relevant statistics about their performance. Since the chromosome being used was simulated, the locations of the repeats in the genome were already known. Therefore, I could determine the number of bases that were accurately and inaccurately categorized as part of a repeat or not part of a repeat (true and false positives or negatives). Such information allowed me to calculate the sensitivity (true positive rate) and specificity (true negative rate) for each tool, in addition to other related calculations.

Creating this evaluation tool gave me experience working with large data sets as well as quantitatively measure the effectiveness of a tool based on its output. Being able to objectively evaluate quality is necessary in order to establish the credibility of a tool, especially when attempting to improve it.

\subsection{Thesis Research}
Repeats can be categorized to be either exact or approximate. An exact repeat is a subsequence that occurs multiple times in the same genome; an approximate repeat is a subsequence that approximately matches multiple other subsequences in the same genome, where two subsequences can differ to some degree and still be considered a valid match. Over long periods of time, originally exactly identical repetitive sequences, which were all copies of some ancestral sequence, have accumulated mutations. Therefore, the detection of approximate repeats can provide information about a particular genome as well as the composition of genomes from previous generations. A repeat-finding algorithm's ability to detect approximate repeats is extremely valuable.

RAIDER, like some other repeat-finding tools, uses spaced seeds to improve the identification of approximate repeats. Spaced seeds are basically patterns describing what positions in two strings must match and what positions in two strings can be a match or a mismatch (don't care positions). RAIDER currently only allows for a single spaced seed to be used in repeat detection, and this seed is chosen arbitarily \cite{figueroa2013raiderpaper}.

Through my thesis research, I hope to improve upon RAIDER's current ability to detect approximate repeats through an in-depth study of spaced seeds. I am currently investigating whether the sensitivity of RAIDER to the detection of approximate repeats could be improved through the use of multiple seeds, more careful seed design and analysis, and/or changes to the algorithm that would make it more amenable to the use of spaced seeds. 
